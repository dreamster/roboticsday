\documentclass[compress]{beamer}

\usepackage[spanish]{babel}
\usepackage[utf8]{inputenc}
\usepackage{amsfonts}
\usepackage{amsmath}
\usepackage{graphicx}
% \usepackage{subfigure}
\usepackage{fancyvrb}
%\usepackage{multicolumn}

\mode<presentation>
{
	\usetheme{Warsaw}
	\setbeamertemplate{headline}{}
	\setbeamertemplate{navigation symbols}{}
	\setbeamercovered{transparent}
	\useinnertheme{rectangles}
}
\setcounter{tocdepth}{1}

\title[Dreamster Workshop]{Dreamster Workshop @ RoboticsDay 2015}
\author[Robotics Day]
{
The Dreamster Community \\
Lucas, Lucía, Sebastián, Viviana, Martín, Gisela, Ignacio
}

\usepackage{default}

\begin{document}

\frame{\titlepage}

\section{Introducción}
\subsection{Objetivos}

\begin{frame}
 \frametitle{¿Qué vamos a hacer?}

\begin{center}
 \color{red}{\textbf{Limpiar un escritorio con un robot}}
\end{center}

\begin{block}{¿Cómo?}
\begin{itemize}
 \item Usando un Dreamster como base.
 \item Arduino para ensenarle a hacer cosas.
\end{itemize}
\end{block}

\begin{center}
\includegraphics[width=0.8\textwidth]{./img/desk.png} 
\end{center}

\end{frame}
 
\begin{frame}
\begin{center}
\frametitle{¿Qué vamos a hacer?}
\includegraphics[width=0.6\textwidth]{./img/equipo.jpg}

\color{red}{\textbf{Trabajar en equipos}}
\end{center}
\begin{itemize}
 \item Cada equipo va a tener un robot
 \item Hoy y mañana preparamos los robots.
 \item El domingo hacemos una comptencia.
\end{itemize}

\end{frame}

\subsection{Presentando el Arduino}

% Arduino fiscamente
\begin{frame}
 \frametitle{¿Qué es Arduino?}
\begin{block}{Arduino.cc dice:} 
\begin{itemize}
 \item Computadora que nos permite interactuar con el mundo real.
 \item Utiliza un microcontrolador simple y un entorno de desarrollo sencillo.
\end{itemize}
\end{block}
\begin{center}
 \includegraphics[height=0.6\textheight]{./img/arduino-leonardo-pinout.png}
\end{center}
\end{frame}

% Programando arduino
\begin{frame}
 \frametitle{¿Qué es Arduino?}
\begin{columns}
 \column{0.6\textwidth}
  \includegraphics[width=\columnwidth]{./img/arduino-ide-text.png}
\column{0.5\textwidth}
  \begin{itemize}
   \item Multiplataforma
   \item Incluye un editor de código
   \item consola serie
   \item transfiere el código a la placa
  \end{itemize}

\end{columns}

\end{frame}

\subsection{Presentando el Dreamster}
\begin{frame}
 \frametitle{¿Qué es Dreamster?}
 \begin{center}
 \includegraphics[width=0.8\textwidth]{./img/dreamster_logo.png}
 \vspace{0.5cm}
 
 \color{red}{\textbf{¡Veamos un video!}}
 \end{center}
\end{frame}

\begin{frame}
 \frametitle{El robot}
 \begin{center}
 \includegraphics[height=0.8\textheight]{./img/dreamster-top.jpg} 
 \end{center}
\end{frame}

\begin{frame}
 \frametitle{El robot}
 \begin{center}
 \includegraphics[height=0.8\textheight]{./img/dreamster-back.jpg} 
 \end{center}
\end{frame}

% La competencia
\begin{frame}
 \frametitle{La competencia}
Tareas:
\begin{itemize}
 \item Encontrar un objeto
 \item Moverse en la dirección del objeto
 \item Identificar el borde de la mesa
\end{itemize}
Qué podemos hacer:
\begin{itemize}
 \item Medir distancias
 \item Medir el color del piso
 \item Movernos
\end{itemize}
 \end{frame}


\section{Programación del Dreamster}
\subsection{Uso del compilador y IO}

\begin{frame}[fragile]
 \frametitle{Primeros pasos}
\begin{columns}
 \column{0.5\textwidth}
\begin{enumerate}
 \item Usuarios registrados.

Nombre de usuario: \verb|taller_mbed_0X|.

Por ej.: \verb|taller_mbed_01|

Clave: \verb|taller_mbed_0X|
 \item Entrar al compilador.
\end{enumerate}
\column{0.5\textwidth}
\begin{center}
 \includegraphics[width=\columnwidth]{./img/web.png}
 % web.png: 993x736 pixel, 98dpi, 25.74x19.08 cm, bb=0 0 730 541
\end{center}
\end{columns}
\end{frame}

\begin{frame}[fragile]
 \frametitle{Nuestro primer código}
Vamos a prender un led $\rightarrow$ Creamos un programa nuevo.
\begin{Verbatim}[formatcom=\color{red},fontseries=b, numbers=left,numbersep=3pt]
 DigitalOut myled(LED1);
          ...
 myled = 1;
\end{Verbatim} 
Muy fácil.... usemos el botón para prender el led..
\begin{Verbatim}[formatcom=\color{red},fontseries=b, numbers=left,numbersep=3pt]
 DigitalIn boton(p18);
          ...
 boton.mode(PullUp);
 if (boton == 0) {
  wait(0.01);     //debounce.
  if (boton == 0) //hacer algo.
          ...
\end{Verbatim} 
\end{frame}

\subsection{Timers y PWM}
\begin{frame}[fragile]
 \frametitle{Complicándola un poco}
Dimmer para los leds.
\begin{columns}
\column{0.6\textwidth}
\begin{description}
 \item[PWM:] Pulse Width Modulation, modulación de ancho de pulso.
 \item[Ticker:] Interfaz usada para definir un evento que se repite a intervalos regulares de tiempo.
 \end{description}
\column{0.4\textwidth}
\begin{center}
 \includegraphics[width=\textwidth]{./img/pwm_image.pdf}
 % pwm_image.pdf: 166x91 pixel, 72dpi, 5.86x3.21 cm, bb=0 0 166 91
\end{center}
\end{columns}

\begin{center}
\begin{Verbatim}[formatcom=\color{red},fontseries=b, numbers=left,numbersep=3pt]
 Ticker evento;
 PwmOut led1(LED1);
      ...
 evento.attach(&dimmer, 0.1);
      ...
 led1 = led1 + 0.1;
\end{Verbatim}
\end{center}
\end{frame}

\subsection{Puerto Serie}
\begin{frame}[fragile]
 \frametitle{¡Hola mundo!}
Todos lo estábamos esperando...
\begin{Verbatim}[formatcom=\color{red},fontseries=b, numbers=left,numbersep=3pt]
  Serial pc(USBTX, USBRX); // tx, rx
            ...
  pc.printf("Hello World!\n"); 
\end{Verbatim}
\begin{center}
 \includegraphics[width=0.6\textwidth]{./img/AccessPort.jpg}
 % AccessPort.jpg: 772x512 pixel, 762dpi, 2.57x1.71 cm, bb=0 0 73 48
\end{center}
\end{frame}

\begin{frame}
 \frametitle{Intervalo!!}
\vspace{2cm}
\begin{center}
\Huge{Nos vemos en 15 minutos.}
\end{center}
\vspace{2cm}
\end{frame}

\begin{frame}
 \frametitle{Bienvenidos}

Los autitos los pudimos hacer gracias a la ayuda del Club de Robótica y la rama IEEE de FIUBA.
\begin{columns}
\column{.5\textwidth}
\begin{center}
 \includegraphics[width=\columnwidth]{./img/robotica.png}
 % robotica.png: 2326x2340 pixel, 90dpi, 65.65x66.05 cm, bb=0 0 1861 1872
\end{center}
\column{.5\textwidth}
\begin{center}
 \includegraphics[width=\columnwidth]{./img/ieee.png}
 % robotica.png: 2326x2340 pixel, 90dpi, 65.65x66.05 cm, bb=0 0 1861 1872
\end{center}
\end{columns}

\end{frame}

\section{Conociendo la planta}
\subsection{Vista general}
\begin{frame}
 \frametitle{El mbed Car - top}
\begin{center}
 \includegraphics[height=0.8\textheight]{./img/mbed_car_top.png}
 % mbed_car_top.pdf: 2147x1792 pixel, 72dpi, 75.74x63.22 cm, bb=0 0 2147 1792
\end{center}
% Foto del auto con las partes marcadas
\end{frame}

\begin{frame}
 \frametitle{El mbed Car - bottom}
\begin{center}
 \includegraphics[height=0.8\textheight]{./img/mbed_car_bottom.png}
 % mbed_car_top.pdf: 2147x1792 pixel, 72dpi, 75.74x63.22 cm, bb=0 0 2147 1792
\end{center}
% Foto del auto con las partes marcadas
\end{frame}


\subsection{Actuadores y sensores}
\begin{frame}[fragile]
 \frametitle{Sensores - TCRT1000}
\begin{columns}
\column{0.7\textwidth}
La salida del sensor es un valor analógico $0 \sim 3.3V$, conectado al AD del mbed. Usando \verb|AnalogIn::read()| se ve 
como un valor float entre 0 y 1.
\begin{center}
 \includegraphics[width=0.3\columnwidth]{./img/cny70.jpg}
 % cny70.jpg: 250x238 pixel, 180dpi, 3.53x3.36 cm, bb=0 0 100 95
\end{center}
\column{0.3\textwidth}
\begin{center}
 \includegraphics[width=0.8\columnwidth]{./img/sensor.pdf}
 % sensor.pdf: 81x195 pixel, 72dpi, 2.86x6.88 cm, bb=0 0 81 195
\end{center}
\end{columns}
\end{frame}

\begin{frame}
\frametitle{Sensores - CNY70 ¿Y cómo veo la linea?}
\begin{center}
 \includegraphics[height=0.8\textheight]{./img/sensores_image.pdf}
 % pwm_image.pdf: 166x91 pixel, 72dpi, 5.86x3.21 cm, bb=0 0 166 91
\end{center}
\end{frame}

\begin{frame}
 \frametitle{Puente H - DRV8833}
Para manejar los motores de CC usamos un IC (DRV8833) que tiene 2 puentes H.
\begin{columns}
\column{0.6\textwidth}
\begin{center}
 \includegraphics[width=\columnwidth]{./img/puenteH_1.png}
 % puenteH_1.png: 1000x614 pixel, 72dpi, 35.28x21.66 cm, bb=0 0 1000 614
\end{center}
\column{0.4\textwidth}
\begin{center}
 \includegraphics[width=\columnwidth]{./img/l298.jpg}
 % puenteH_1.png: 1000x614 pixel, 72dpi, 35.28x21.66 cm, bb=0 0 1000 614
\end{center}

\end{columns}
\end{frame}

\begin{frame}
 \frametitle{Puente H - DRV8833}
Con el puente H podemos cambiar el sentido y velocidad de los motores.
\begin{center}
 \includegraphics[width=\columnwidth]{./img/puenteH_2.png}
 % puenteH_1.png: 1000x614 pixel, 72dpi, 35.28x21.66 cm, bb=0 0 1000 614
\end{center}
.. y ¿Cómo variamos la velocidad?

\end{frame}

\begin{frame}
 \frametitle{Puente H - DRV8833}
% diagrama de un puente H del 298, con los enables y los IN#.
% tablita de la hoja de datos.
Pero en realidad, no lo vamos a usar así. Uso de los puentes H integrados en el L298.
\begin{columns} 
\column{0.5\textwidth}
{%
\begin{center}
\begin{tabular}{|c|c|c|l|}\hline
\textbf{EN} & \textbf{IN1} & \textbf{IN2} & \multicolumn{1}{c|}{\textbf{Motor}}\\\hline
0 & X & X & Libre\\\hline
1 & 0 & 0 & Frenado\\\hline
1 & 1 & 0 & Adelante\\\hline
1 & 0 & 1 & Atrás\\\hline
1 & 1 & 1 & Frenado\\\hline
\end{tabular}
\end{center}
}%
\column{0.5\textwidth}
 \begin{center}
 \includegraphics[width=\columnwidth]{./img/l298image.png}
 % l298image.png: 322x353 pixel, 100dpi, 8.18x8.97 cm, bb=0 0 232 254
\end{center}
\end{columns}
\end{frame}

\begin{frame}
\frametitle{A trabajar...}
Con todo lo que ya vimos, ahora sólo necesitan saber donde está conectada cada cosa :)
\begin{center}
\begin{tabular}{|c|c|c|c|}
\hline
\textbf{Función} & \textbf{PinName} & \textbf{Función} & \textbf{PinName}\\\hline
Farol 1 & p9 & PH1 IN1 & p28\\\hline
Farol 2 & p10 & PH1 IN2 & p27\\\hline
Led Sensor 1 & p14 & PH1 EN & p25\\\hline
Sensor 1 & p16 & PH2 IN1 & p30\\\hline
Led Sensor 2 & p13 & PH2 IN2 & p29\\\hline
Sensor 2 & p17 & PH2 EN & p26\\\hline
Botón & p18 & & \\\hline
\end{tabular}
\end{center}
\end{frame}


\end{document}


